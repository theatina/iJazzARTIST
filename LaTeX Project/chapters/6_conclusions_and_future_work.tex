\chapter{CONCLUSIONS AND FUTURE WORK} \label{chapter:conclusions_futurework}
    \section{Conclusions}
    This work, has carried out research on the task of real-time jazz improvisation, a collaborative process between a human and artificial agent, based on predefined chord charts, through implicit machine learning techniques. The main purpose of the thesis was to study the employment of deep NNs for the aforementioned matter, the ability thereof to model the expectation and its violation in jazz improvisation scenarios and the artificial agent's ability to develop a model of expectation for the intentions of the human soloist. Additionally, data refinements have been applied to obtain greater variability of the accompaniment of the pieces in the dataset. 

    Testing of the system was performed within two simulated cases of real-time scenarios, accompaniment generation over a random and no solo. The resulting responses under the two jazz standard settings described in \hyperref[sec:jazzStandards]{\textbf{Jazz Standards}} section, have indicated the largely achieved compliance of the system with each chord chart, disregarding the first few time steps of the process, during which the system was considered to be building up memory. The aforementioned behaviour can possibly be attributed to the LSTMs' operational structure and their random initialisation states. Moreover, in the case of ``All of me", the system proved to be unaffected by the simulated human solo, as well as characterised by repetition, in most accompaniment sessions. In contrast, ``Au Privave" has demonstrated greater dependence on the randomly constructed solo, conclusion drawn from the demonstrated decrease in self-repetition and increase in the variability of voicings over particular chart chords.  

    \section{Future Work}
    %In order for the real-time accompaniment system to be studied more profoundly in terms of the proposed task, further research is imperative. 
    The results of the study indicate the capability of modeling the violation of expectation for jazz accompaniment in a real-time scenario by employing deep NNs. However, some limitations have emerged, concerning the dataset as well as the computational complexity. 

    The lack of proper datasets that incorporate all the information needed for the task (lead sheet chords, metric information, solo and accompaniment), raised the need for data construction or refinement and integration thereof to the initial dataset. For instance, data enrichment was performed, aiming to increase the dataset's variability, using rudimentary probabilistic methods, which in some cases renders the transitions inconsistent, difficult to be learnt from the system. 

    Regarding the music style, the most common genre found in the datasets available up to this point is pop, which is characterised by less variability compared to jazz music. Therefore, a dataset based on jazz standard accompaniment is of utmost importance for proper examination of the discussed hypothesis. 

    Within the scope of real-time application of the developed system, the execution time also has proven to be marginally acceptable for performing the task. In order to comply with the real-time setting to a degree, the time resolution that was applied to the dataset, has constrained the system in terms of expressive capabilities both for capturing the human agent's characteristics and for creating the accompaniment for the solo.
