\abstractEn{


Some of the most essential characteristics of improvisation on jazz standards are reflected through the accompaniment. Given a lead sheet as common ground, the study of the collaborative process of music improvisation between a human and an artificial agent in a real time setting, is a scenario of great interest in the MIR domain. So far, the approaches concerning the jazz improvisation accompaniment procedure, have presented systems that lack the capability of performing the accompaniment generation task while at the same time adapting to dynamically variable constraints depending on new, improvised data. The thesis at hand, proposes a jazz accompaniment system capable of providing proper chord voicings to the solo, while complying with both the soloist's intentions as well as the previously defined constraints set by the lead sheet. The artificial agent consists of two sub-systems; a model responsible for predicting the human soloist's intentions and a second system performing the task of the accompaniment. The latter is achieved by modeling the artificial agent's predictions, after exploiting the information on the expectations of the human agent's intentions, previously calculated by the first model. Recurrent Neural Networks (RNNs) comprise both aforementioned models. The dataset used in the training process has undergone multi-staged processing including probabilistic refinement, aiming to keep and enrich the information which is requisite for the task. The system was tested on two cases of jazz standards, demonstrating ability of compliance with the harmonic constraints. Additionally, output variability depending on the solo improvisation has been indicated. Emerging limitations as well as potential future perspectives are discussed in the conclusion of this work.


}