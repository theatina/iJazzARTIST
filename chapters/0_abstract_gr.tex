\abstractGr{

\begin{greek}

Κάποια από τα κυριότερα χαρακτηριστικά του αυτοσχεδιασμού σε πρότυπα τζαζ εκφράζονται μέσα από τη μουσική συνοδεία. Η συνεργασία μεταξύ ανθρώπων και τεχνητών συστημάτων για την επίτευξη αυτοσχεδιασμού σε πραγματικό χρόνο, υπό το πλαίσιο κοινής παρτιτούρας, αποτελεί ένα ιδιαίτερα ενδιαφέρον αντικείμενο μελέτης για τον τομέα της Ανάκτησης Μουσικής Πληροφορίας. Οι προϋπάρχουσες προσεγγίσεις που αφορούν στη διαδικασία της συνοδείας τζαζ αυτοσχεδιασμού, έχουν παρουσιάσει συστήματα που δε διαθέτουν την ικανότητα συμμόρφωσης με δυναμικά μεταβαλλόμενα περιβάλλοντα, εξαρτώμενα από τα αυτοσχέδια δεδομένα. Η παρούσα πτυχιακή εργασία παρουσιάζει ένα σύστημα συνοδείας, το οποίο διαθέτει την ικανότητα προσαρμογής τόσο στο τζαζ σόλο του μουσικού, όσο και τους περιορισμούς που έχουν προκαθοριστεί από την παρτιτούρα. Ο τεχνητός πράκτορας που αναπτύσσεται για το σκοπό αυτό, αποτελείται από δύο υποσυστήματα \textperiodcentered \ ένα μοντέλο υπεύθυνο για την παραγωγή προβλέψεων που αφορούν το σόλο του μουσικού κι ένα δεύτερο υποσύστημα που παράγει την τελική μουσική συνοδεία, αξιοποιώντας την πληροφορία για τις προθέσεις του σολίστα που παρήγαγε το πρώτο μοντέλο. Και τα δύο προαναφερθέντα μοντέλα έχουν ως σχεδιαστική βάση τα Αναδρομικά Νευρωνικά Δίκτυα. Το σύνολο των δεδομένων που χρησιμοποιήθηκαν στην εκπαίδευση των μοντέλων υποβλήθηκαν σε επεξεργασία πολλών επιπέδων, συμπεριλαμβανομένης της πιθανολογικής βελτιστοποίησης, με στόχο τη διατήρηση και την επαύξηση της χρήσιμης πληροφορίας. Το τελικό σύστημα εξετάστηκε με τη χρήση δύο τζαζ προτύπων, παρουσιάζοντας προσαρμοστική ικανότητα ως προς τους αρμονικούς περιορισμούς, καθώς και ποικιλομορφία, εξαρτώμενη από τον αυτοσχεδιασμό του μουσικού. Τέλος, αναφέρονται κάποιες δυσκολίες που προέκυψαν, όπως επίσης και προτάσεις για περαιτέρω έρευνα. 

\end{greek}

}